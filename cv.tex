\documentclass[a4paper,11pt]{article}
\usepackage[francais]{babel}  
\usepackage[utf8]{inputenc}   
\usepackage[T1]{fontenc}       
\usepackage{url}
\pagestyle{empty}    
\usepackage{vmargin} 
\setmarginsrb{2cm}{2cm}{2cm}{2cm}{0cm}{0cm}{0cm}{0cm}
% Marge gauche, haute, droite, basse; espace entre la marge et le texte à
% gauche, en  haut, à droite, en bas
\newcommand\espace{\vrule height 20pt width 0pt}
\newcommand{\titre}[1]{%
	\begin{center}
	\bigskip
	\rule{\textwidth}{1pt}
	\par\vspace{0.1cm}
        \textbf{\LARGE #1}
	\par\rule{\textwidth}{1pt}
	\end{center}
	\bigskip
}

%###################
\begin{document}
%###################
\begin{center}
\par\textbf{\huge Curriculum Vitae}
\end{center}

\vspace{1.5cm}


\begin{minipage}{0.48\linewidth}
\begin{flushleft}
Anthonin Bonnefoy\\
\end{flushleft}
\end{minipage}
\hfill
\begin{minipage}{0.48\linewidth}
\begin{flushright}
Nationalité française \\
23 ans, célibataire
\end{flushright}
\end{minipage}

\titre{Formation}
%#############

\begin{tabular}{c@{ }p{0.8\textwidth}}

\textbf{2006--2009} &  École d'ingénieurs \textbf{ISIMA} (\textbf{I}nstitut \textbf{S}upérieur d'\textbf{I}nformatique, de \textbf{M}odélisation et de leurs \textbf{A}pplications) à Clermont-Ferrand (63)\\
& Option : Génie logiciel et Systèmes Informatiques\\

\textbf{2004--2006} &  Classe préparatoire aux grandes école\\
&  Filières \textbf{PSI} à Champollion, Grenoble (38)\\

\textbf{2003-2006} &  Baccalauréat \textbf{S}, mention bien

\end{tabular}

\titre{Compétences}
\begin{itemize} 
\item \textbf{Systèmes} : Linux, Unix, BSD 
\item \textbf{Développement} : Java, Scala, Haskell, Lua
\item \textbf{Frameworks Java} : Spring, Hibernate, Google Web Toolkit, PicoContainer
\item \textbf{Gestion de version} : Git, Subversion
\item \textbf{Base de données} : MySQL, PostgreSQL
\item \textbf{Langues} : Anglais : TOEIC 920\\
          Espagnol : notions\\
          Japonais : notions
\end{itemize}

\titre{Expériences et projets}

\begin{tabular}{l@{ }p{0.8\textwidth}}
Projet 2010 & \textbf{Despote sur IzPack}\\
&Refonte et refactoring du code pour la version 5 \\
&Ajout de tests automatiques \\
&Remplacement de Ant par Maven pour la gestion du build\\
&Introduction d'un framework d'injection de dépendance\\
& \textit{Cadre} : Projet Open Source (Fondation Codehaus) \\
& \textit{Outils} : IntelliJ, Java, Git, PicoContainer, Fest-Swing \\
\espace
Emploi 2009-2010 & \textbf{Ingénieur d'étude} \\
& Travail d'expertise sur les tests automatiques \\
& Support sur l'utilisation de maven et de la forge logicielle \\
& \textit{Cadre} : Orange Business Services \\
\espace
Stage 2009 & \textbf{Études des outils de tests Java et de la forge logicielle} \\
& Évaluation des outils pour des tests unitaires, d'intégrations, fonctionnelles et de performances \\
& Étude, amélioration et installation de forge logicielle \\
& Rédaction d'un support de formation pour Maven2 \\
& \textit{Cadre} : Orange Business Services \\
& \textit{Outils} : Maven, Junit, TestNG, Mockito, dbunit, Selenium, Fest-swing, JBehave, JMeter\\
\espace 
Projet 2008/2009 & \textbf{Refactoring de IzPack} \\ 
& Suppression de la dépendance a nanoXML \\
& \textit{Cadre} : Projet Open Source \\
& \textit{Outils} : IntelliJ, Java, Git, Methode agile \\
\espace
Stage 2008 & \textbf{Participation a l'élaboration du site de Météo France}  \\
& Développement de brique de l'espace service \\ 
& \textit{Cadre} : Atos WorldLine, Lyon (69) \\
& \textit{Outils} : Spring, Tapestry, Hibernate, Maven2, Integration Continue \\ 
\espace
Projet 2007/2008 & \textbf{Étude du framework Google Web Toolkit} \\
& Avantages et inconvénients du framework \\
& Application de l'étude à la création d'un agenda Web dynamique \\ 
& \textit{Cadre} : Reflex Training (63) \\
& \textit{Outils} : Eclipse, Java, GWT \\

\end{tabular}

\titre{Expériences diverses}
\begin{tabular}{ll}
Vie associative :& Président du \textbf{Rezzo}, club organisant des soirées réseaux a l'ISIMA \\
& Vice-secrétaire du \textbf{LVDIC}, club de jeux de sociétés et de jeux de rôles \\
Sport pratiqué :& Basket (6 ans)
\end{tabular}

\end{document}

